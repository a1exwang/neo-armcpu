% $File: ucore.tex
% $Date: Fri Jan 31 21:02:49 2014 +0800
% $Author: jiakai <jia.kai66@gmail.com>

\section{ucore}
\subsection{编译}
如果使用较新的gcc(4.7.3)编译,ucore中的一个bug会暴露出来:
编译\verb|kern/trap/vector.S|时,编译器会自动在跳转语句后插入NOP,
导致所有的偏移量出错,解决方法是在文件头部加入\verb|.set noreorder|。
然而,即使修复了这个bug,编译出的代码仍然不能在我们的CPU上正常运行。
最终我们使用了与最初移植ucore到mips的学长所用的相同的版本的gcc\footnote{%
\url{https://sourcery.mentor.com/GNUToolchain/release2189}},
编译出的代码可正常运行。

另外,我们尝试了打开gcc的-O2优化,发现系统运行速度明显提高,
但系统引导时会触发\verb|fs/sfs/sfs_fs.c|中的
\verb|panic("unused_blocks not equal!\n")|。我们未能找到其原因,
后来注释掉这条panic发现系统可以正常工作,于是均使用-O2优化了。


\subsection{使用qemu模拟}
修改ucore时,如果能在真实硬件上运行前先用模拟器模拟,将能极大提高开发效率。
好在移植ucore的学长已经修改了qemu模拟器,可于\url{%
https://github.com/chyh1990/qemu-thumips}下载,并使用
\verb|./configure --target-list=mipsel-softmmu|编译;根据系统的不同,
可能需要手动修复一些链接错误。


\subsection{抓取远程文件}
为了实现从PC上抓取文件并在本地执行,需要实现一个抓取远程文件的系统调用。
为此,首先修改了控制台驱动,当输出一个字符时,会先往串口写入一个MAGIC0字符,
再写入实际输出的字符;另外增加了抓取文件的系统调用,
它被调用时会先向串口写入MAGIC1字符,接下来进入抓取文件相关的通信协议。
同时需要修改PC上的终端程序,让它遇到MAGIC0时就显示接下来的字符,
遇到MAGIC1时就进入文件传输模式。另外由于ucore的VFS不支持新建文件,
因此只能先在ramdisk中预留好一个足够大的空文件,把抓取到的文件内容写入这个其中。


\subsection{VGA输出}
为了实现控制台内容同步输出到VGA,首先需要字体文件。
我们通过一个脚本渲染了某字体的可见ascii字符,并将得到的位图写入了一个C数组中。
在ucore中加入了简单的字体渲染模块,需要处理换行、滚屏等,同时修改控制台驱动,
每输出一个字符,在向串口输出的同时,也需要将该字符传递给字体渲染模块。

另外,为了能让用户程序进行VGA绘图后正常返回控制台,
还提供了重绘全屏幕的系统调用。由于ucore没有提供物理内存映射的相关调用,
同时也为了简化设计,CPU中没有处理访问特权内存的异常,
这样用户程序可以通过直接访问VGA缓存来操作屏幕内容。


% vim: filetype=tex foldmethod=marker foldmarker=f{{{,f}}}

